\documentclass[pdf]{beamer}
\usepackage{lmodern}
\mode<presentation>{\usetheme{PaloAlto}}
\title{Speckle Interferometry}
\author{Matt Rauen and David Fan}
\date{\today}
\begin{document}
\begin{frame}
\titlepage
\end{frame}

\section{Abstract}
\begin{frame}{Abstract}

\end{frame}

\section{Research Questions}
\begin{frame}{Research Questions}
\begin{center}
Primary Question:

How does the piezo deform when a voltage is applied to it?
\end{center}
More concretely, we want to construct a topological map of the shift at each point, where each point is characterized by the phase difference (relative to the wavelength of light we used) between its original position and its position after the applied voltage.
\begin{figure}[htbp]
\centering
\includegraphics[width=0.3\textwidth]{gaussian_bump.png}
\end{figure}
\end{frame}

\section{Background}
\begin{frame}{Speckle Patterns}

\end{frame}

\begin{frame}{Modeling Each Speckle}

\end{frame}

\section{Setup}
\begin{frame}{Setup}

\end{frame}

\section{Analysis}
\begin{frame}{Calculating Absolute Phases}

\end{frame}

\begin{frame}{Calculating Absolute Phases}

\end{frame}

\begin{frame}{Calculating Relative Phases}

\end{frame}

\section{Results}
\begin{frame}{Results}

\end{frame}

\section{Conclusions}
\begin{frame}{Conclusions}

\end{frame}

\section{Future Research}
\begin{frame}{Future Research}

\end{frame}
\end{document}